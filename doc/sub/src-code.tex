% gebruik:
% % gebruik:
% % gebruik:
% % gebruik:
% \input{src-code} %eventueel: \input{src-code.tex}
%
% ------------------------------------------------------------------------------

%
% -------- voorkeuren --------------------------------------
\usepackage{listings}%http://ctan.cs.uu.nl/macros/latex/contrib/listings/listings.pdf
\usepackage{xcolor}

%
% ---- instellingen ----------------------------------------
\definecolor{mygreen}{rgb}{0,0.6,0}
\definecolor{mygray}{rgb}{0.5,0.5,0.5}
\definecolor{mymauve}{rgb}{0.58,0,0.82}

\lstset{ 
  backgroundcolor=\color{white},  % package color of xcolor nodig na listings
  basicstyle=\footnotesize\sffamily,% the size of the fonts that are used for the code
  breakatwhitespace=false,        % sets if automatic breaks should only happen at whitespace
  breaklines=false,               % sets automatic line breaking
  captionpos=b,                   % sets the caption-position to bottom
  commentstyle=\color{mygreen},   % comment style
  deletekeywords={...},           % if you want to delete keywords from the given language
  escapeinside={\%*}{*)},         % if you want to add LaTeX within your code
  extendedchars=true,             % non-ASCII characters; 8-bits encodings only, no UTF-8
  firstnumber=1000,               % start line enumeration with line 1000
  frame=single,	                  % adds a frame around the code
  keepspaces=true,                % keeps indentation of code (possibly needs columns=flexible)
  keywordstyle=\color{blue},      % keyword style
  language=Octave,                % the language of the code
  morekeywords={*,...},           % if you want to add more keywords to the set
  numbers=none,                   % line-numbers; possible are (none, left, right)
  numbersep=5pt,                  % how far the line-numbers are from the code
  numberstyle=\tiny\color{mygray},% the style that is used for the line-numbers
  rulecolor=\color{black},        % if not set, the frame-color may be changed on line-breaks within not-black text (e.g. comments (green here))
  showspaces=false,               % show spaces adding underscores; overrides 'showstringspaces'
  showstringspaces=false,         % underline spaces within strings only
  showtabs=false,                 % show tabs within strings adding particular underscores
  stepnumber=2,                   % the step between line-numbers. 1 numbers each line
  stringstyle=\color{mymauve},    % string literal style
  tabsize=4,	                  % sets default tabsize to 4 spaces
  title=\lstname                  % show name of file included with \lstinputlisting; also try caption instead of title
}
 %eventueel: % gebruik:
% \input{src-code} %eventueel: \input{src-code.tex}
%
% ------------------------------------------------------------------------------

%
% -------- voorkeuren --------------------------------------
\usepackage{listings}%http://ctan.cs.uu.nl/macros/latex/contrib/listings/listings.pdf
\usepackage{xcolor}

%
% ---- instellingen ----------------------------------------
\definecolor{mygreen}{rgb}{0,0.6,0}
\definecolor{mygray}{rgb}{0.5,0.5,0.5}
\definecolor{mymauve}{rgb}{0.58,0,0.82}

\lstset{ 
  backgroundcolor=\color{white},  % package color of xcolor nodig na listings
  basicstyle=\footnotesize\sffamily,% the size of the fonts that are used for the code
  breakatwhitespace=false,        % sets if automatic breaks should only happen at whitespace
  breaklines=false,               % sets automatic line breaking
  captionpos=b,                   % sets the caption-position to bottom
  commentstyle=\color{mygreen},   % comment style
  deletekeywords={...},           % if you want to delete keywords from the given language
  escapeinside={\%*}{*)},         % if you want to add LaTeX within your code
  extendedchars=true,             % non-ASCII characters; 8-bits encodings only, no UTF-8
  firstnumber=1000,               % start line enumeration with line 1000
  frame=single,	                  % adds a frame around the code
  keepspaces=true,                % keeps indentation of code (possibly needs columns=flexible)
  keywordstyle=\color{blue},      % keyword style
  language=Octave,                % the language of the code
  morekeywords={*,...},           % if you want to add more keywords to the set
  numbers=none,                   % line-numbers; possible are (none, left, right)
  numbersep=5pt,                  % how far the line-numbers are from the code
  numberstyle=\tiny\color{mygray},% the style that is used for the line-numbers
  rulecolor=\color{black},        % if not set, the frame-color may be changed on line-breaks within not-black text (e.g. comments (green here))
  showspaces=false,               % show spaces adding underscores; overrides 'showstringspaces'
  showstringspaces=false,         % underline spaces within strings only
  showtabs=false,                 % show tabs within strings adding particular underscores
  stepnumber=2,                   % the step between line-numbers. 1 numbers each line
  stringstyle=\color{mymauve},    % string literal style
  tabsize=4,	                  % sets default tabsize to 4 spaces
  title=\lstname                  % show name of file included with \lstinputlisting; also try caption instead of title
}

%
% ------------------------------------------------------------------------------

%
% -------- voorkeuren --------------------------------------
\usepackage{listings}%http://ctan.cs.uu.nl/macros/latex/contrib/listings/listings.pdf
\usepackage{xcolor}

%
% ---- instellingen ----------------------------------------
\definecolor{mygreen}{rgb}{0,0.6,0}
\definecolor{mygray}{rgb}{0.5,0.5,0.5}
\definecolor{mymauve}{rgb}{0.58,0,0.82}

\lstset{ 
  backgroundcolor=\color{white},  % package color of xcolor nodig na listings
  basicstyle=\footnotesize\sffamily,% the size of the fonts that are used for the code
  breakatwhitespace=false,        % sets if automatic breaks should only happen at whitespace
  breaklines=false,               % sets automatic line breaking
  captionpos=b,                   % sets the caption-position to bottom
  commentstyle=\color{mygreen},   % comment style
  deletekeywords={...},           % if you want to delete keywords from the given language
  escapeinside={\%*}{*)},         % if you want to add LaTeX within your code
  extendedchars=true,             % non-ASCII characters; 8-bits encodings only, no UTF-8
  firstnumber=1000,               % start line enumeration with line 1000
  frame=single,	                  % adds a frame around the code
  keepspaces=true,                % keeps indentation of code (possibly needs columns=flexible)
  keywordstyle=\color{blue},      % keyword style
  language=Octave,                % the language of the code
  morekeywords={*,...},           % if you want to add more keywords to the set
  numbers=none,                   % line-numbers; possible are (none, left, right)
  numbersep=5pt,                  % how far the line-numbers are from the code
  numberstyle=\tiny\color{mygray},% the style that is used for the line-numbers
  rulecolor=\color{black},        % if not set, the frame-color may be changed on line-breaks within not-black text (e.g. comments (green here))
  showspaces=false,               % show spaces adding underscores; overrides 'showstringspaces'
  showstringspaces=false,         % underline spaces within strings only
  showtabs=false,                 % show tabs within strings adding particular underscores
  stepnumber=2,                   % the step between line-numbers. 1 numbers each line
  stringstyle=\color{mymauve},    % string literal style
  tabsize=4,	                  % sets default tabsize to 4 spaces
  title=\lstname                  % show name of file included with \lstinputlisting; also try caption instead of title
}
 %eventueel: % gebruik:
% % gebruik:
% \input{src-code} %eventueel: \input{src-code.tex}
%
% ------------------------------------------------------------------------------

%
% -------- voorkeuren --------------------------------------
\usepackage{listings}%http://ctan.cs.uu.nl/macros/latex/contrib/listings/listings.pdf
\usepackage{xcolor}

%
% ---- instellingen ----------------------------------------
\definecolor{mygreen}{rgb}{0,0.6,0}
\definecolor{mygray}{rgb}{0.5,0.5,0.5}
\definecolor{mymauve}{rgb}{0.58,0,0.82}

\lstset{ 
  backgroundcolor=\color{white},  % package color of xcolor nodig na listings
  basicstyle=\footnotesize\sffamily,% the size of the fonts that are used for the code
  breakatwhitespace=false,        % sets if automatic breaks should only happen at whitespace
  breaklines=false,               % sets automatic line breaking
  captionpos=b,                   % sets the caption-position to bottom
  commentstyle=\color{mygreen},   % comment style
  deletekeywords={...},           % if you want to delete keywords from the given language
  escapeinside={\%*}{*)},         % if you want to add LaTeX within your code
  extendedchars=true,             % non-ASCII characters; 8-bits encodings only, no UTF-8
  firstnumber=1000,               % start line enumeration with line 1000
  frame=single,	                  % adds a frame around the code
  keepspaces=true,                % keeps indentation of code (possibly needs columns=flexible)
  keywordstyle=\color{blue},      % keyword style
  language=Octave,                % the language of the code
  morekeywords={*,...},           % if you want to add more keywords to the set
  numbers=none,                   % line-numbers; possible are (none, left, right)
  numbersep=5pt,                  % how far the line-numbers are from the code
  numberstyle=\tiny\color{mygray},% the style that is used for the line-numbers
  rulecolor=\color{black},        % if not set, the frame-color may be changed on line-breaks within not-black text (e.g. comments (green here))
  showspaces=false,               % show spaces adding underscores; overrides 'showstringspaces'
  showstringspaces=false,         % underline spaces within strings only
  showtabs=false,                 % show tabs within strings adding particular underscores
  stepnumber=2,                   % the step between line-numbers. 1 numbers each line
  stringstyle=\color{mymauve},    % string literal style
  tabsize=4,	                  % sets default tabsize to 4 spaces
  title=\lstname                  % show name of file included with \lstinputlisting; also try caption instead of title
}
 %eventueel: % gebruik:
% \input{src-code} %eventueel: \input{src-code.tex}
%
% ------------------------------------------------------------------------------

%
% -------- voorkeuren --------------------------------------
\usepackage{listings}%http://ctan.cs.uu.nl/macros/latex/contrib/listings/listings.pdf
\usepackage{xcolor}

%
% ---- instellingen ----------------------------------------
\definecolor{mygreen}{rgb}{0,0.6,0}
\definecolor{mygray}{rgb}{0.5,0.5,0.5}
\definecolor{mymauve}{rgb}{0.58,0,0.82}

\lstset{ 
  backgroundcolor=\color{white},  % package color of xcolor nodig na listings
  basicstyle=\footnotesize\sffamily,% the size of the fonts that are used for the code
  breakatwhitespace=false,        % sets if automatic breaks should only happen at whitespace
  breaklines=false,               % sets automatic line breaking
  captionpos=b,                   % sets the caption-position to bottom
  commentstyle=\color{mygreen},   % comment style
  deletekeywords={...},           % if you want to delete keywords from the given language
  escapeinside={\%*}{*)},         % if you want to add LaTeX within your code
  extendedchars=true,             % non-ASCII characters; 8-bits encodings only, no UTF-8
  firstnumber=1000,               % start line enumeration with line 1000
  frame=single,	                  % adds a frame around the code
  keepspaces=true,                % keeps indentation of code (possibly needs columns=flexible)
  keywordstyle=\color{blue},      % keyword style
  language=Octave,                % the language of the code
  morekeywords={*,...},           % if you want to add more keywords to the set
  numbers=none,                   % line-numbers; possible are (none, left, right)
  numbersep=5pt,                  % how far the line-numbers are from the code
  numberstyle=\tiny\color{mygray},% the style that is used for the line-numbers
  rulecolor=\color{black},        % if not set, the frame-color may be changed on line-breaks within not-black text (e.g. comments (green here))
  showspaces=false,               % show spaces adding underscores; overrides 'showstringspaces'
  showstringspaces=false,         % underline spaces within strings only
  showtabs=false,                 % show tabs within strings adding particular underscores
  stepnumber=2,                   % the step between line-numbers. 1 numbers each line
  stringstyle=\color{mymauve},    % string literal style
  tabsize=4,	                  % sets default tabsize to 4 spaces
  title=\lstname                  % show name of file included with \lstinputlisting; also try caption instead of title
}

%
% ------------------------------------------------------------------------------

%
% -------- voorkeuren --------------------------------------
\usepackage{listings}%http://ctan.cs.uu.nl/macros/latex/contrib/listings/listings.pdf
\usepackage{xcolor}

%
% ---- instellingen ----------------------------------------
\definecolor{mygreen}{rgb}{0,0.6,0}
\definecolor{mygray}{rgb}{0.5,0.5,0.5}
\definecolor{mymauve}{rgb}{0.58,0,0.82}

\lstset{ 
  backgroundcolor=\color{white},  % package color of xcolor nodig na listings
  basicstyle=\footnotesize\sffamily,% the size of the fonts that are used for the code
  breakatwhitespace=false,        % sets if automatic breaks should only happen at whitespace
  breaklines=false,               % sets automatic line breaking
  captionpos=b,                   % sets the caption-position to bottom
  commentstyle=\color{mygreen},   % comment style
  deletekeywords={...},           % if you want to delete keywords from the given language
  escapeinside={\%*}{*)},         % if you want to add LaTeX within your code
  extendedchars=true,             % non-ASCII characters; 8-bits encodings only, no UTF-8
  firstnumber=1000,               % start line enumeration with line 1000
  frame=single,	                  % adds a frame around the code
  keepspaces=true,                % keeps indentation of code (possibly needs columns=flexible)
  keywordstyle=\color{blue},      % keyword style
  language=Octave,                % the language of the code
  morekeywords={*,...},           % if you want to add more keywords to the set
  numbers=none,                   % line-numbers; possible are (none, left, right)
  numbersep=5pt,                  % how far the line-numbers are from the code
  numberstyle=\tiny\color{mygray},% the style that is used for the line-numbers
  rulecolor=\color{black},        % if not set, the frame-color may be changed on line-breaks within not-black text (e.g. comments (green here))
  showspaces=false,               % show spaces adding underscores; overrides 'showstringspaces'
  showstringspaces=false,         % underline spaces within strings only
  showtabs=false,                 % show tabs within strings adding particular underscores
  stepnumber=2,                   % the step between line-numbers. 1 numbers each line
  stringstyle=\color{mymauve},    % string literal style
  tabsize=4,	                  % sets default tabsize to 4 spaces
  title=\lstname                  % show name of file included with \lstinputlisting; also try caption instead of title
}

%
% ------------------------------------------------------------------------------

%
% -------- voorkeuren --------------------------------------
\usepackage{listings}%http://ctan.cs.uu.nl/macros/latex/contrib/listings/listings.pdf
\usepackage{xcolor}

%
% ---- instellingen ----------------------------------------
\definecolor{mygreen}{rgb}{0,0.6,0}
\definecolor{mygray}{rgb}{0.5,0.5,0.5}
\definecolor{mymauve}{rgb}{0.58,0,0.82}

\lstset{ 
  backgroundcolor=\color{white},  % package color of xcolor nodig na listings
  basicstyle=\footnotesize\sffamily,% the size of the fonts that are used for the code
  breakatwhitespace=false,        % sets if automatic breaks should only happen at whitespace
  breaklines=false,               % sets automatic line breaking
  captionpos=b,                   % sets the caption-position to bottom
  commentstyle=\color{mygreen},   % comment style
  deletekeywords={...},           % if you want to delete keywords from the given language
  escapeinside={\%*}{*)},         % if you want to add LaTeX within your code
  extendedchars=true,             % non-ASCII characters; 8-bits encodings only, no UTF-8
  firstnumber=1000,               % start line enumeration with line 1000
  frame=single,	                  % adds a frame around the code
  keepspaces=true,                % keeps indentation of code (possibly needs columns=flexible)
  keywordstyle=\color{blue},      % keyword style
  language=Octave,                % the language of the code
  morekeywords={*,...},           % if you want to add more keywords to the set
  numbers=none,                   % line-numbers; possible are (none, left, right)
  numbersep=5pt,                  % how far the line-numbers are from the code
  numberstyle=\tiny\color{mygray},% the style that is used for the line-numbers
  rulecolor=\color{black},        % if not set, the frame-color may be changed on line-breaks within not-black text (e.g. comments (green here))
  showspaces=false,               % show spaces adding underscores; overrides 'showstringspaces'
  showstringspaces=false,         % underline spaces within strings only
  showtabs=false,                 % show tabs within strings adding particular underscores
  stepnumber=2,                   % the step between line-numbers. 1 numbers each line
  stringstyle=\color{mymauve},    % string literal style
  tabsize=4,	                  % sets default tabsize to 4 spaces
  title=\lstname                  % show name of file included with \lstinputlisting; also try caption instead of title
}
 %eventueel: % gebruik:
% % gebruik:
% % gebruik:
% \input{src-code} %eventueel: \input{src-code.tex}
%
% ------------------------------------------------------------------------------

%
% -------- voorkeuren --------------------------------------
\usepackage{listings}%http://ctan.cs.uu.nl/macros/latex/contrib/listings/listings.pdf
\usepackage{xcolor}

%
% ---- instellingen ----------------------------------------
\definecolor{mygreen}{rgb}{0,0.6,0}
\definecolor{mygray}{rgb}{0.5,0.5,0.5}
\definecolor{mymauve}{rgb}{0.58,0,0.82}

\lstset{ 
  backgroundcolor=\color{white},  % package color of xcolor nodig na listings
  basicstyle=\footnotesize\sffamily,% the size of the fonts that are used for the code
  breakatwhitespace=false,        % sets if automatic breaks should only happen at whitespace
  breaklines=false,               % sets automatic line breaking
  captionpos=b,                   % sets the caption-position to bottom
  commentstyle=\color{mygreen},   % comment style
  deletekeywords={...},           % if you want to delete keywords from the given language
  escapeinside={\%*}{*)},         % if you want to add LaTeX within your code
  extendedchars=true,             % non-ASCII characters; 8-bits encodings only, no UTF-8
  firstnumber=1000,               % start line enumeration with line 1000
  frame=single,	                  % adds a frame around the code
  keepspaces=true,                % keeps indentation of code (possibly needs columns=flexible)
  keywordstyle=\color{blue},      % keyword style
  language=Octave,                % the language of the code
  morekeywords={*,...},           % if you want to add more keywords to the set
  numbers=none,                   % line-numbers; possible are (none, left, right)
  numbersep=5pt,                  % how far the line-numbers are from the code
  numberstyle=\tiny\color{mygray},% the style that is used for the line-numbers
  rulecolor=\color{black},        % if not set, the frame-color may be changed on line-breaks within not-black text (e.g. comments (green here))
  showspaces=false,               % show spaces adding underscores; overrides 'showstringspaces'
  showstringspaces=false,         % underline spaces within strings only
  showtabs=false,                 % show tabs within strings adding particular underscores
  stepnumber=2,                   % the step between line-numbers. 1 numbers each line
  stringstyle=\color{mymauve},    % string literal style
  tabsize=4,	                  % sets default tabsize to 4 spaces
  title=\lstname                  % show name of file included with \lstinputlisting; also try caption instead of title
}
 %eventueel: % gebruik:
% \input{src-code} %eventueel: \input{src-code.tex}
%
% ------------------------------------------------------------------------------

%
% -------- voorkeuren --------------------------------------
\usepackage{listings}%http://ctan.cs.uu.nl/macros/latex/contrib/listings/listings.pdf
\usepackage{xcolor}

%
% ---- instellingen ----------------------------------------
\definecolor{mygreen}{rgb}{0,0.6,0}
\definecolor{mygray}{rgb}{0.5,0.5,0.5}
\definecolor{mymauve}{rgb}{0.58,0,0.82}

\lstset{ 
  backgroundcolor=\color{white},  % package color of xcolor nodig na listings
  basicstyle=\footnotesize\sffamily,% the size of the fonts that are used for the code
  breakatwhitespace=false,        % sets if automatic breaks should only happen at whitespace
  breaklines=false,               % sets automatic line breaking
  captionpos=b,                   % sets the caption-position to bottom
  commentstyle=\color{mygreen},   % comment style
  deletekeywords={...},           % if you want to delete keywords from the given language
  escapeinside={\%*}{*)},         % if you want to add LaTeX within your code
  extendedchars=true,             % non-ASCII characters; 8-bits encodings only, no UTF-8
  firstnumber=1000,               % start line enumeration with line 1000
  frame=single,	                  % adds a frame around the code
  keepspaces=true,                % keeps indentation of code (possibly needs columns=flexible)
  keywordstyle=\color{blue},      % keyword style
  language=Octave,                % the language of the code
  morekeywords={*,...},           % if you want to add more keywords to the set
  numbers=none,                   % line-numbers; possible are (none, left, right)
  numbersep=5pt,                  % how far the line-numbers are from the code
  numberstyle=\tiny\color{mygray},% the style that is used for the line-numbers
  rulecolor=\color{black},        % if not set, the frame-color may be changed on line-breaks within not-black text (e.g. comments (green here))
  showspaces=false,               % show spaces adding underscores; overrides 'showstringspaces'
  showstringspaces=false,         % underline spaces within strings only
  showtabs=false,                 % show tabs within strings adding particular underscores
  stepnumber=2,                   % the step between line-numbers. 1 numbers each line
  stringstyle=\color{mymauve},    % string literal style
  tabsize=4,	                  % sets default tabsize to 4 spaces
  title=\lstname                  % show name of file included with \lstinputlisting; also try caption instead of title
}

%
% ------------------------------------------------------------------------------

%
% -------- voorkeuren --------------------------------------
\usepackage{listings}%http://ctan.cs.uu.nl/macros/latex/contrib/listings/listings.pdf
\usepackage{xcolor}

%
% ---- instellingen ----------------------------------------
\definecolor{mygreen}{rgb}{0,0.6,0}
\definecolor{mygray}{rgb}{0.5,0.5,0.5}
\definecolor{mymauve}{rgb}{0.58,0,0.82}

\lstset{ 
  backgroundcolor=\color{white},  % package color of xcolor nodig na listings
  basicstyle=\footnotesize\sffamily,% the size of the fonts that are used for the code
  breakatwhitespace=false,        % sets if automatic breaks should only happen at whitespace
  breaklines=false,               % sets automatic line breaking
  captionpos=b,                   % sets the caption-position to bottom
  commentstyle=\color{mygreen},   % comment style
  deletekeywords={...},           % if you want to delete keywords from the given language
  escapeinside={\%*}{*)},         % if you want to add LaTeX within your code
  extendedchars=true,             % non-ASCII characters; 8-bits encodings only, no UTF-8
  firstnumber=1000,               % start line enumeration with line 1000
  frame=single,	                  % adds a frame around the code
  keepspaces=true,                % keeps indentation of code (possibly needs columns=flexible)
  keywordstyle=\color{blue},      % keyword style
  language=Octave,                % the language of the code
  morekeywords={*,...},           % if you want to add more keywords to the set
  numbers=none,                   % line-numbers; possible are (none, left, right)
  numbersep=5pt,                  % how far the line-numbers are from the code
  numberstyle=\tiny\color{mygray},% the style that is used for the line-numbers
  rulecolor=\color{black},        % if not set, the frame-color may be changed on line-breaks within not-black text (e.g. comments (green here))
  showspaces=false,               % show spaces adding underscores; overrides 'showstringspaces'
  showstringspaces=false,         % underline spaces within strings only
  showtabs=false,                 % show tabs within strings adding particular underscores
  stepnumber=2,                   % the step between line-numbers. 1 numbers each line
  stringstyle=\color{mymauve},    % string literal style
  tabsize=4,	                  % sets default tabsize to 4 spaces
  title=\lstname                  % show name of file included with \lstinputlisting; also try caption instead of title
}
 %eventueel: % gebruik:
% % gebruik:
% \input{src-code} %eventueel: \input{src-code.tex}
%
% ------------------------------------------------------------------------------

%
% -------- voorkeuren --------------------------------------
\usepackage{listings}%http://ctan.cs.uu.nl/macros/latex/contrib/listings/listings.pdf
\usepackage{xcolor}

%
% ---- instellingen ----------------------------------------
\definecolor{mygreen}{rgb}{0,0.6,0}
\definecolor{mygray}{rgb}{0.5,0.5,0.5}
\definecolor{mymauve}{rgb}{0.58,0,0.82}

\lstset{ 
  backgroundcolor=\color{white},  % package color of xcolor nodig na listings
  basicstyle=\footnotesize\sffamily,% the size of the fonts that are used for the code
  breakatwhitespace=false,        % sets if automatic breaks should only happen at whitespace
  breaklines=false,               % sets automatic line breaking
  captionpos=b,                   % sets the caption-position to bottom
  commentstyle=\color{mygreen},   % comment style
  deletekeywords={...},           % if you want to delete keywords from the given language
  escapeinside={\%*}{*)},         % if you want to add LaTeX within your code
  extendedchars=true,             % non-ASCII characters; 8-bits encodings only, no UTF-8
  firstnumber=1000,               % start line enumeration with line 1000
  frame=single,	                  % adds a frame around the code
  keepspaces=true,                % keeps indentation of code (possibly needs columns=flexible)
  keywordstyle=\color{blue},      % keyword style
  language=Octave,                % the language of the code
  morekeywords={*,...},           % if you want to add more keywords to the set
  numbers=none,                   % line-numbers; possible are (none, left, right)
  numbersep=5pt,                  % how far the line-numbers are from the code
  numberstyle=\tiny\color{mygray},% the style that is used for the line-numbers
  rulecolor=\color{black},        % if not set, the frame-color may be changed on line-breaks within not-black text (e.g. comments (green here))
  showspaces=false,               % show spaces adding underscores; overrides 'showstringspaces'
  showstringspaces=false,         % underline spaces within strings only
  showtabs=false,                 % show tabs within strings adding particular underscores
  stepnumber=2,                   % the step between line-numbers. 1 numbers each line
  stringstyle=\color{mymauve},    % string literal style
  tabsize=4,	                  % sets default tabsize to 4 spaces
  title=\lstname                  % show name of file included with \lstinputlisting; also try caption instead of title
}
 %eventueel: % gebruik:
% \input{src-code} %eventueel: \input{src-code.tex}
%
% ------------------------------------------------------------------------------

%
% -------- voorkeuren --------------------------------------
\usepackage{listings}%http://ctan.cs.uu.nl/macros/latex/contrib/listings/listings.pdf
\usepackage{xcolor}

%
% ---- instellingen ----------------------------------------
\definecolor{mygreen}{rgb}{0,0.6,0}
\definecolor{mygray}{rgb}{0.5,0.5,0.5}
\definecolor{mymauve}{rgb}{0.58,0,0.82}

\lstset{ 
  backgroundcolor=\color{white},  % package color of xcolor nodig na listings
  basicstyle=\footnotesize\sffamily,% the size of the fonts that are used for the code
  breakatwhitespace=false,        % sets if automatic breaks should only happen at whitespace
  breaklines=false,               % sets automatic line breaking
  captionpos=b,                   % sets the caption-position to bottom
  commentstyle=\color{mygreen},   % comment style
  deletekeywords={...},           % if you want to delete keywords from the given language
  escapeinside={\%*}{*)},         % if you want to add LaTeX within your code
  extendedchars=true,             % non-ASCII characters; 8-bits encodings only, no UTF-8
  firstnumber=1000,               % start line enumeration with line 1000
  frame=single,	                  % adds a frame around the code
  keepspaces=true,                % keeps indentation of code (possibly needs columns=flexible)
  keywordstyle=\color{blue},      % keyword style
  language=Octave,                % the language of the code
  morekeywords={*,...},           % if you want to add more keywords to the set
  numbers=none,                   % line-numbers; possible are (none, left, right)
  numbersep=5pt,                  % how far the line-numbers are from the code
  numberstyle=\tiny\color{mygray},% the style that is used for the line-numbers
  rulecolor=\color{black},        % if not set, the frame-color may be changed on line-breaks within not-black text (e.g. comments (green here))
  showspaces=false,               % show spaces adding underscores; overrides 'showstringspaces'
  showstringspaces=false,         % underline spaces within strings only
  showtabs=false,                 % show tabs within strings adding particular underscores
  stepnumber=2,                   % the step between line-numbers. 1 numbers each line
  stringstyle=\color{mymauve},    % string literal style
  tabsize=4,	                  % sets default tabsize to 4 spaces
  title=\lstname                  % show name of file included with \lstinputlisting; also try caption instead of title
}

%
% ------------------------------------------------------------------------------

%
% -------- voorkeuren --------------------------------------
\usepackage{listings}%http://ctan.cs.uu.nl/macros/latex/contrib/listings/listings.pdf
\usepackage{xcolor}

%
% ---- instellingen ----------------------------------------
\definecolor{mygreen}{rgb}{0,0.6,0}
\definecolor{mygray}{rgb}{0.5,0.5,0.5}
\definecolor{mymauve}{rgb}{0.58,0,0.82}

\lstset{ 
  backgroundcolor=\color{white},  % package color of xcolor nodig na listings
  basicstyle=\footnotesize\sffamily,% the size of the fonts that are used for the code
  breakatwhitespace=false,        % sets if automatic breaks should only happen at whitespace
  breaklines=false,               % sets automatic line breaking
  captionpos=b,                   % sets the caption-position to bottom
  commentstyle=\color{mygreen},   % comment style
  deletekeywords={...},           % if you want to delete keywords from the given language
  escapeinside={\%*}{*)},         % if you want to add LaTeX within your code
  extendedchars=true,             % non-ASCII characters; 8-bits encodings only, no UTF-8
  firstnumber=1000,               % start line enumeration with line 1000
  frame=single,	                  % adds a frame around the code
  keepspaces=true,                % keeps indentation of code (possibly needs columns=flexible)
  keywordstyle=\color{blue},      % keyword style
  language=Octave,                % the language of the code
  morekeywords={*,...},           % if you want to add more keywords to the set
  numbers=none,                   % line-numbers; possible are (none, left, right)
  numbersep=5pt,                  % how far the line-numbers are from the code
  numberstyle=\tiny\color{mygray},% the style that is used for the line-numbers
  rulecolor=\color{black},        % if not set, the frame-color may be changed on line-breaks within not-black text (e.g. comments (green here))
  showspaces=false,               % show spaces adding underscores; overrides 'showstringspaces'
  showstringspaces=false,         % underline spaces within strings only
  showtabs=false,                 % show tabs within strings adding particular underscores
  stepnumber=2,                   % the step between line-numbers. 1 numbers each line
  stringstyle=\color{mymauve},    % string literal style
  tabsize=4,	                  % sets default tabsize to 4 spaces
  title=\lstname                  % show name of file included with \lstinputlisting; also try caption instead of title
}

%
% ------------------------------------------------------------------------------

%
% -------- voorkeuren --------------------------------------
\usepackage{listings}%http://ctan.cs.uu.nl/macros/latex/contrib/listings/listings.pdf
\usepackage{xcolor}

%
% ---- instellingen ----------------------------------------
\definecolor{mygreen}{rgb}{0,0.6,0}
\definecolor{mygray}{rgb}{0.5,0.5,0.5}
\definecolor{mymauve}{rgb}{0.58,0,0.82}

\lstset{ 
  backgroundcolor=\color{white},  % package color of xcolor nodig na listings
  basicstyle=\footnotesize\sffamily,% the size of the fonts that are used for the code
  breakatwhitespace=false,        % sets if automatic breaks should only happen at whitespace
  breaklines=false,               % sets automatic line breaking
  captionpos=b,                   % sets the caption-position to bottom
  commentstyle=\color{mygreen},   % comment style
  deletekeywords={...},           % if you want to delete keywords from the given language
  escapeinside={\%*}{*)},         % if you want to add LaTeX within your code
  extendedchars=true,             % non-ASCII characters; 8-bits encodings only, no UTF-8
  firstnumber=1000,               % start line enumeration with line 1000
  frame=single,	                  % adds a frame around the code
  keepspaces=true,                % keeps indentation of code (possibly needs columns=flexible)
  keywordstyle=\color{blue},      % keyword style
  language=Octave,                % the language of the code
  morekeywords={*,...},           % if you want to add more keywords to the set
  numbers=none,                   % line-numbers; possible are (none, left, right)
  numbersep=5pt,                  % how far the line-numbers are from the code
  numberstyle=\tiny\color{mygray},% the style that is used for the line-numbers
  rulecolor=\color{black},        % if not set, the frame-color may be changed on line-breaks within not-black text (e.g. comments (green here))
  showspaces=false,               % show spaces adding underscores; overrides 'showstringspaces'
  showstringspaces=false,         % underline spaces within strings only
  showtabs=false,                 % show tabs within strings adding particular underscores
  stepnumber=2,                   % the step between line-numbers. 1 numbers each line
  stringstyle=\color{mymauve},    % string literal style
  tabsize=4,	                  % sets default tabsize to 4 spaces
  title=\lstname                  % show name of file included with \lstinputlisting; also try caption instead of title
}

%
% ------------------------------------------------------------------------------

%
% -------- voorkeuren --------------------------------------
\usepackage{listings}%http://ctan.cs.uu.nl/macros/latex/contrib/listings/listings.pdf
\usepackage{xcolor}

%
% ---- instellingen ----------------------------------------
\definecolor{mygreen}{rgb}{0,0.6,0}
\definecolor{mygray}{rgb}{0.5,0.5,0.5}
\definecolor{mymauve}{rgb}{0.58,0,0.82}

\lstset{ 
  backgroundcolor=\color{white},  % package color of xcolor nodig na listings
  basicstyle=\footnotesize\sffamily,% the size of the fonts that are used for the code
  breakatwhitespace=false,        % sets if automatic breaks should only happen at whitespace
  breaklines=false,               % sets automatic line breaking
  captionpos=b,                   % sets the caption-position to bottom
  commentstyle=\color{mygreen},   % comment style
  deletekeywords={...},           % if you want to delete keywords from the given language
  escapeinside={\%*}{*)},         % if you want to add LaTeX within your code
  extendedchars=true,             % non-ASCII characters; 8-bits encodings only, no UTF-8
  firstnumber=1000,               % start line enumeration with line 1000
  frame=single,	                  % adds a frame around the code
  keepspaces=true,                % keeps indentation of code (possibly needs columns=flexible)
  keywordstyle=\color{blue},      % keyword style
  language=Octave,                % the language of the code
  morekeywords={*,...},           % if you want to add more keywords to the set
  numbers=none,                   % line-numbers; possible are (none, left, right)
  numbersep=5pt,                  % how far the line-numbers are from the code
  numberstyle=\tiny\color{mygray},% the style that is used for the line-numbers
  rulecolor=\color{black},        % if not set, the frame-color may be changed on line-breaks within not-black text (e.g. comments (green here))
  showspaces=false,               % show spaces adding underscores; overrides 'showstringspaces'
  showstringspaces=false,         % underline spaces within strings only
  showtabs=false,                 % show tabs within strings adding particular underscores
  stepnumber=2,                   % the step between line-numbers. 1 numbers each line
  stringstyle=\color{mymauve},    % string literal style
  tabsize=4,	                  % sets default tabsize to 4 spaces
  title=\lstname                  % show name of file included with \lstinputlisting; also try caption instead of title
}
